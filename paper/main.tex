\documentclass{article}
\usepackage{arxiv}

\usepackage[utf8]{inputenc}
\usepackage[english, russian]{babel}
\usepackage[T1]{fontenc}
\usepackage{url}
\usepackage{booktabs}
\usepackage{amsfonts}
\usepackage{nicefrac}
\usepackage{microtype}
\usepackage{lipsum}
\usepackage{graphicx}
\usepackage{natbib}
\usepackage{doi}



\title{A template for the \emph{arxiv} style}

\author{ David S.~Hippocampus\thanks{Use footnote for providing further
		information about author (webpage, alternative
		address)---\emph{not} for acknowledging funding agencies.} \\
	Department of Computer Science\\
	Cranberry-Lemon University\\
	Pittsburgh, PA 15213 \\
	\texttt{hippo@cs.cranberry-lemon.edu} \\
	%% examples of more authors
	\And
	Elias D.~Striatum \\
	Department of Electrical Engineering\\
	Mount-Sheikh University\\
	Santa Narimana, Levand \\
	\texttt{stariate@ee.mount-sheikh.edu} \\
	%% \AND
	%% Coauthor \\
	%% Affiliation \\
	%% Address \\
	%% \texttt{email} \\
	%% \And
	%% Coauthor \\
	%% Affiliation \\
	%% Address \\
	%% \texttt{email} \\
	%% \And
	%% Coauthor \\
	%% Affiliation \\
	%% Address \\
	%% \texttt{email} \\
}
\date{}

\renewcommand{\shorttitle}{\textit{arXiv} Template}

%%% Add PDF metadata to help others organize their library
%%% Once the PDF is generated, you can check the metadata with
%%% $ pdfinfo template.pdf
\hypersetup{
pdftitle={A template for the arxiv style},
pdfsubject={q-bio.NC, q-bio.QM},
pdfauthor={David S.~Hippocampus, Elias D.~Striatum},
pdfkeywords={First keyword, Second keyword, More},
}

\begin{document}
\maketitle

\begin{abstract}
	\lipsum[1]
\end{abstract}


\keywords{First keyword \and Second keyword \and More}

\section{Introduction}
Благодаря улучшению структурированности рассуждений и фактологической точности генераций большие языковые модели (LLM) стали повсеместным инструментом в научных рабочих процессах, таких как предметно-ориентированный вопросно-ответный поиск, суммаризация и подготовка научных обзоров. Объём и тематическое разнообразие научных публикаций неуклонно растут, продолжая отдаляться от пределов того, что отдельные исследователи или малые группы способны анализировать на регулярной основе, мотивируя развитие методов автоматизации решения этих задач. Подготовка качественных обзоров на таком материале остаётся технически сложной для существующих решений: она требует качественного извлечения информации, выделения главных методов и результатов, атрибуции, основанной на корректном цитировании, и устойчивой работы в условиях сверхдлинного контекста. В данной работе предлагается провести исследование эффективности в описанных условиях агентных решений.

Ранние подходы к задаче многодокументной суммаризации заключались в применении генеративных трансформерных моделей, однако по мере роста требований к размерам корпусов и длине контекстов они быстро эволюционировали в сторону RAG, где механизм извлечения релевантных фрагментов заземляет генерацию на фиксированном наборе источников и повышает фактологическую точность [1]. При этом в практике RAG широко используются сложные многоитерационные пайплайны, вручную спроектированные исследователями и во многом опирающиеся на экспертную интуицию, что предопределяет последовательность стадий извлечения, отбора и генерации [1]. Параллельно предпринимаются попытки применять RL к составлению обзоров: варьируются способы построения сигналов качества и агрегирования предпочтений (включая обучение по человеческим оценкам и по правкам текста), причём наибольшую практическую отдачу демонстрирует RLHF, а при дефиците разметки — его масштабируемая альтернатива RLAIF [1]. Дальнейшее развитие идей пошагового планирования проявилось в схемах ReAct и многоагентных конвейерах «writer–critic», применяемых к близким по характеру задачам длинного контекста; по сравнению с жёстко заданными RAG-скриптами такие решения обеспечивают большую гибкость распределения ролей и согласования промежуточных выводов [1]. Наконец, в задачах веб-поиска эффективность показали методы дообучения агентных систем с использованием RL, что подтверждает перспективность аналогичных стратегий для обзорной генерации [1].

Несмотря на существенный прогресс, текущие решения остаются ограниченными. RAG‑пайплайны задают жёсткую последовательность шагов и плохо моделируют межстатейные связи, поэтому при росте корпуса дают фрагментарные, слабо таксономизированные «обзоры». Агентные подходы применялись в основном к специфическим задачам (таблицы, длинные нарративы) и не рассматривались как носители RL‑обучения; между тем именно они предоставляют естественный каркас для обучения политики взаимодействия с инструментами и памятью. В задачах сверхдлинного контекста критически важен способ переноса и сжатия рабочего контекста между итерациями (working memory), однако наиболее результативные методы краткосрочной памяти  не применялись к построению научных обзоров. Работы по RL для суммаризации преимущественно оптимизируют прокси‑метрики и почти не интегрируются с многоагентной координацией, что ограничивает перенос на научные обзоры. Наконец, существующие системы преимущественно решают задачу поиска, тогда как главные узкие места лежат в обработке, агрегировании и проверяемой атрибуции информации.

Мы разрабатывали агентную систему, адаптированную под задачу составления научных обзоров. Её ядро составляют два взаимодействующих агента — автор и критик, координирующих генерацию и проверку текста. Автор формирует черновой обзор, используя специализированные инструменты для работы с источниками: семантический поиск по фиксированному корпусу, извлечение цитат, переформулирование фрагментов и управление краткосрочной памятью, которая агрегирует промежуточные суждения и помогает поддерживать когерентность между разделами. Критик анализирует высказывания автора, проверяя их на соответствие исходным статьям, выявляет неатрибутированные утверждения также опираясь на рабочую или эпизодическую память. Мы реализовали  среду для дальнейшего обучения с подкреплением, включающую: инструменты дл яработы с фиксированным корпусом текстов; подсистему памяти; протокол пошагового взаимодействия между двумя агентами с трассировкой действий и снимками памяти для воспроизводимости; различные функцийи награды (оценивающие покрытие, фактологичность, организационную согласованность) и сценарии оценки.

Мы представили новую постановку задачи генерации обзора, исключив компоненту поиска, что позволило изолированно исследовать способности агента к анализу и агрегации информации. Мы показали, что компактные модели, работающие в агентном режиме, достигают качества, сопоставимого с существенно более крупными моделями на тех же задачах генерации обзоров. Анализ влияния модулей памяти показал, что его применение критически важно для решения рассматриваемой проблемы и дает значительный прирост метрик фактологической точности генерации и ее связности. Наконец, мы представляем фреймворк с реализацией агентной среды для составления обзоров по фиксированным корпусам документов для последующих исследований и воспроизводимого обучения.
\section{Headings: first level}
\label{sec:headings}

\lipsum[4] See Section \ref{sec:headings}.

\subsection{Headings: second level}
\lipsum[5]
\begin{equation}
	\xi _{ij}(t)=P(x_{t}=i,x_{t+1}=j|y,v,w;\theta)= {\frac {\alpha _{i}(t)a^{w_t}_{ij}\beta _{j}(t+1)b^{v_{t+1}}_{j}(y_{t+1})}{\sum _{i=1}^{N} \sum _{j=1}^{N} \alpha _{i}(t)a^{w_t}_{ij}\beta _{j}(t+1)b^{v_{t+1}}_{j}(y_{t+1})}}
\end{equation}

\subsubsection{Headings: third level}
\lipsum[6]

\paragraph{Paragraph}
\lipsum[7]



\section{Examples of citations, figures, tables, references}
\label{sec:others}

\subsection{Citations}
Citations use \verb+natbib+. The documentation may be found at
\begin{center}
	\url{http://mirrors.ctan.org/macros/latex/contrib/natbib/natnotes.pdf}
\end{center}

Here is an example usage of the two main commands (\verb+citet+ and \verb+citep+): Some people thought a thing \citep{kour2014real, hadash2018estimate} but other people thought something else \citep{kour2014fast}. Many people have speculated that if we knew exactly why \citet{kour2014fast} thought this\dots

\subsection{Figures}
\lipsum[10]
See Figure \ref{fig:fig1}. Here is how you add footnotes. \footnote{Sample of the first footnote.}
\lipsum[11]

\begin{figure}
	\centering
	\includegraphics[width=0.5\textwidth]{../figures/log_reg_cs_exp.eps}
	\caption{Sample figure caption.}
	\label{fig:fig1}
\end{figure}

\subsection{Tables}
See awesome Table~\ref{tab:table}.

The documentation for \verb+booktabs+ (`Publication quality tables in LaTeX') is available from:
\begin{center}
	\url{https://www.ctan.org/pkg/booktabs}
\end{center}


\begin{table}
	\caption{Sample table title}
	\centering
	\begin{tabular}{lll}
		\toprule
		\multicolumn{2}{c}{Part}                   \\
		\cmidrule(r){1-2}
		Name     & Description     & Size ($\mu$m) \\
		\midrule
		Dendrite & Input terminal  & $\sim$100     \\
		Axon     & Output terminal & $\sim$10      \\
		Soma     & Cell body       & up to $10^6$  \\
		\bottomrule
	\end{tabular}
	\label{tab:table}
\end{table}

\subsection{Lists}
\begin{itemize}
	\item Lorem ipsum dolor sit amet
	\item consectetur adipiscing elit.
	\item Aliquam dignissim blandit est, in dictum tortor gravida eget. In ac rutrum magna.
\end{itemize}


\bibliographystyle{unsrtnat}
\bibliography{references}

\end{document}
